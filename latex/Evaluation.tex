% ------------------------------------------------------------------------
% -*-TeX-*- -*-Hard-*- Smart Wrapping
% ------------------------------------------------------------------------
\def\baselinestretch{1}

\chapter{Conclusion}
This chapter explains evaluation, necessary discussion, conclusion summary and future work for my dissertation.

\def\baselinestretch{1.66}

%%% ----------------------------------------------------------------------
 

%%% ----------------------------------------------------------------------
\goodbreak
\section{Evaluation and Discussion}
An evaluation of software is a specific kind of assessment that is carried out with the purpose of determining whether or not a software programme or a collection of software applications is the sort of solution that is most suited to meeting the requirements of a certain customer. The purpose of this exercise is to take a closer look at the tools and resources that are made available by the application that is either being used at the moment by that client or is being considered as a potential addition to the programmes that are now being used by that client.

\subsection{Advantages of Chatbot}

\subsubsection{1. Available at any time:}
I'm sure the vast majority of you are routinely required to wait on hold as technicians link you to a customer care executive. They are taking the place of live chat as well as other, more time-consuming means of communication, such as emails and phone calls. Chatbots are essentially virtual robots, which means that they never grow tired and will continue to follow whatever instruction you give them. Your user experience (UX) will improve as a result, and you will climb the rankings in your industry. You also have the ability to expertly build your chatbot in order to keep your goal and emphasis in the right place, which is another benefit of this fast answer.

\subsubsection{2. Managing capability:}
Chat bots have the capacity to have discussions with thousands of individuals at the same time, in contrast to humans, who can only have a conversation with one other person at a time. Imagine that you are the owner of a restaurant that is well-known for the quality of its cuisine and that the majority of your profits come from delivery orders. You will have a greater number of clients to receive orders from, but you will only have a limited number of staff members to cater to all of these consumers. Having a chatbot will do rid of this issue, make it possible to attend to every single customer, and guarantee that no orders are overlooked. Chatbots are already being used by businesses such as Taco Bell and Domino's to coordinate the delivery of customers' packages.

\subsubsection{3. Characteristic of adaptability:}
The flexibility of chatbots makes them suitable for application in almost any sector because to their low entry barrier. Chatbots, in contrast to other products, for which switching platforms requires a significant amount of development work and testing, are quite simple to implement. It is sufficient to educate the bot by providing it with the appropriate discussion format and flow in order to alter the sector or industry in which it is currently operating.

\subsubsection{4. User appreciation:}
Emotional states fluctuate naturally in humans. Conversely, chatbots must follow the guidelines set forth for them in order to function properly. No matter how rude or profane a customer is, they will always be treated with the utmost professionalism. People's eating preferences might vary from day to day. In this situation, it may learn your address, offer recommendations for future purchases, and more based on your previous purchasing history.

\subsubsection{5. Economical and practical:}
It's never a cheap event to hire a person for a position, which will be considerably more costly if your revenues are not sufficient or sales objectives are not fulfilled, both of which would cause turmoil in the company. Because of the limitations that come with being human, a single person can only effectively interact with one or two other people at any given time. The employee would have a very difficult time if it were any higher than that. Chatbots could be able to assist in finding a solution to this age-old dilemma. Due to the fact that one chatbot is equivalent to a large number of workers, it is readily able to interact with thousands upon thousands of clients all at once. We would just need a small group of individuals to participate in chats here and there when it's really required.

\subsubsection{6. A quicker starting point:}
If you want to get anything done, you need to figure out how to do it first. They won't be qualified for the position unless they can demonstrate this. Every rung up the corporate ladder is accompanied by a new round of training for the individual in question. Not all workers will remain with the company, some will be let go, new ones will be hired, and so on. We seek to convey the reality that workers are subject to change. There will be a significant time investment on the part of existing staff in training the new hires. Chatbots have the potential to drastically reduce this wait time, but only if they follow a predetermined, human-friendly conversational format.

\subsection{Disadvantages of Chatbot}

\subsubsection{1. Inconveniently multi-purpose:}
Many programmers want to make a chatbot that can interact with any platform and function as a true personal assistant. However, practical bots end up not being able to handle the vast majority of questions. They have a number of drawbacks, including a tendency to misunderstand the users, forgetting what they were taught only 5 minutes before, etc. It's hardly surprising given that it takes a team of skilled programmers years to create a global bot that can interpret plain language and assess context. Additionally, even in this scenario, operational initiatives of this kind should be continuously enhanced.

\subsubsection{2. Simplistic Methods}
There are two distinct categories of bots: those that are powered by artificial intelligence and have the ability to pick up new skills via interaction, and those that are pre-programmed to respond appropriately in a variety of situations. Chatbots powered by artificial intelligence are said to be superior because of their ability to react differently based on the circumstances and the setting. However, the creation of intricate algorithms is necessary in order to accomplish this goal. In the meanwhile, only the largest companies in the information technology industry and a select few programmers have such a robust technical basis. As a result, it would be preferable for standard businesses to concentrate on the second kind of bots since they are more dependable and need less complexity. Due to the fact that they are lacking in intellect, it is quite unlikely that they will be able to engage in impolite modes of communication or break free from their creators' authority.

\subsubsection{3. Interface that is sophisticated:}
Conversation with a bot takes place in a chat environment, which requires the user to produce a significant amount of text. And in the event that the bot is unable to comprehend the user's request, it will be necessary for it to speak much more. It takes some time to determine which instructions a bot can answer to appropriately and which inquiries are best avoided in order to maximise its usefulness. Therefore, conversing with a chatbot will, in the vast majority of instances, not result in time savings. It's possible that in the not-too-distant future, virtual assistants may become even more useful thanks to the addition of a speech recognition capability. However, for the time being, its functional capabilities are quite limited, and there are only a select number of business domains in which they may be of any real value.

\subsection{Ethical, Social, and Legal Issues on Information Sharing}
Intelligent assistive technologies have a profound impact on patients’ emotional and psychosocial well-being because of their pervasiveness and ubiquitous nature. Dementia patients may benefit from personalized caring and assisted living technologies that allow them to remain within own homes and carry out daily tasks on their own \citep{leg1}. Correspondingly, owing to the technical uniqueness and complexity of intelligent tachnology for dementia, the incorporation of such systems into standard dementia care poses a series of legal and ethical difficulties. It has been suggested that the transition from humanistic to 13 technology-assisted care may have an unanticipated influence on the individual experience of elderly people living with dementia \citep{leg2}. Additional typical meta-ethical assessments for using intelligent technology in dementia care comprise the proper procurement of expressed permission \citep{leg3} and the safeguarding of individuals’ privacy rights from uninformed monitoring \citep{leg4}.

Although procurement and commissions play a critical role in the development of new ideas, legislative frameworks and inflexible organisational culture frequently lead to a risk-averse and limited approach to strategic purchase in the public sector. In addition, social care AI technology markets are still immature, making the integration of AI into social care delivery a challenging task. As a result of AI, social care may be provided at a lower cost and in a more personalised manner. Risk and reward are important considerations for contracting authority to make informed decisions. When it comes to AI technology and the data it generates, local governments must take additional steps to ensure that any commercial partnership is built on ethical foundations. When it comes to working with the commercial sector, local governments and care providers have a lot of knowledge that may be tapped upon as AI becomes more ubiquitous in social care. It is possible, however, for organisations to take further steps to guarantee that they connect with like-minded partners from the corporate sector, especially in terms of ethics and their commitment to social value.

\subsubsection{Risk factors}
The impact of AI on a broad variety of risk categories has grown in the last two years, including model and compliance risks as well as operational and legal risks as well as reputational and regulatory risks. Fresh and uncharted territory for those who have never used data analysis or model management in their businesses. Despite this, even in companies which possess a tradition of addressing these risks, AI presents these risks in unique and difficult ways. Investigators are still unable to verify the privacy and security of apps for Alzeimer's Disease (AD) or related conditions (AD/RD). Privacy breaches are more likely to occur because of the patient group's cognitive impairment and old age. Future studies will focus on privacy policies and the protection of user data. Because the MARS evaluation was not explicitly intended for AD/RD care–related applications, the researchers were able to capture some difference between both the apps and the MARS assessment. While other AD/RD care applications focus more on functionality, MARS places an emphasis on user involvement.

%\bigskip

%%% ----------------------------------------------------------------------
\goodbreak
\section{Conclusion}
Despite the significant progress that has been made, there is still a long way to go in the creation of intelligent technology that may be utilised in the treatment of dementia. This is the case despite the fact that there has been great progress. There has been a great deal of discourse over the role that assistive technology plays in the delivery of modern healthcare ever since the number of articles, conferences, and workshops that are specifically devoted to this topic has skyrocketed over the course of the last decade. In the years to come, it will be essential for academics and physicians to evaluate the performance attributes and usability of commercial items that originate from these projects in "real world" circumstances. Because of the growing interest among academics in this industry, there has been an increase in the number of national finance mechanisms that specifically investigate the applications of transdisciplinary technology research. However, there is no study of this kind on the construction of intelligent portals for dementia care. This research will aid a significant number of individuals in the healthcare industry, especially those working in dementia care.

%%% ----------------------------------------------------------------------
\goodbreak
\section{Suggestion for Further Work}
Any software, regardless of its current state, can almost always be made better. In the not-too-distant future, we could be expanded to incorporate more functions like File Transfer and Voice Message. It is feasible to make it as user-friendly as is humanly conceivable. It is important to us that it be created for the aim of producing money, and we would also want to give it a web domain. Due to the fact that the code is mostly organised or modular in nature, additional needs and upgrades may be simply implemented. Improvements may be appended by either modifying the modules that are already there or introducing new modules. It is possible to make more improvements to the application so that the website performs in a way that is more appealing to the user and helpful than the current one.

The potential of chatbots is still being explored, but their existing capabilities have their limits. Chatbots are not able to realise their full potential because of the constraints placed on the processing and retrieval of data. It's not for a lack of computer processing capacity on our end; we have enough of that. Nevertheless, there is a restriction on "How" we go about doing it. The market for customers who shop at retail establishments is one of the most prominent examples. Because of the nature of their requirements, clients at retail establishments are most interested in engaging with other people. They do not want their requirements to be analysed by bots and answered accordingly.

