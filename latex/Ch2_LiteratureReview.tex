% ------------------------------------------------------------------------
% -*-TeX-*- -*-Hard-*- Smart Wrapping
% ------------------------------------------------------------------------
\def\baselinestretch{1}

\chapter{Literature Review}

\def\baselinestretch{1.44}

%%% ----------------------------------------------------------------------

This chapter reviews the related literature work in the field of dementia care, as well as looks over chatbot technology in healthcare and natural language processing with knowledge base for my dissertation topic. 
   
\smallskip

%%% ----------------------------------------------------------------------
\goodbreak
\section{Related Work on Dementia}
People with dementia and their caregivers are becoming more interested in creating and testing technology that might enhance their quality of life and care. The Family Caregiving Institute hosted a Research Priorities in Caregiving Summit: Advancing Family-Centered Care Across the Serious Illness Spectrum in 2018. "Evaluate technologies that promote choice and collaborative decision making" and "identify where technology is best incorporated throughout the trajectory of caring" are the first two research goals listed \citep{three}. Such a technological breakthrough might have a significant impact on the field of dementia care. There are several explanations for this. 

Firstly, Delaying or averting its need for brief care in the rapidly ageing population is a first step in alleviating the pressure on public funds and ensuring the viability of institutional services amid a quickly expanding senior group \citep{lit1}. Secondly, another benefit of a large-scale deployment of robotics-assisted care is that it might minimise the load on unpaid carers while also improving quality of care due to a decline in the ratio of caregivers to patients \citep{int1}. Thirdly, and without any viable treatment options in sight, big data platforms can uncover insights from enormous volumes of unstructured data and enhance prevention, diagnosis and therapy and care administration \citep{lit2}. Finally, even more importantly, artificial intelligence (AI) is a powerful tool that may be used to enhance the delivery of patient-centered healthcare solutions that are tailored to an individual's needs \citep{lit3}. In addition to helping patients fulfil their desires, this would empower them and enhance their quality of life. 

According to previous studies, persons with dementia may utilize a variety of technologies tailored to their requirements, such as computers \citep{four} and touchscreen devices \citep{five}. Likewise, efforts to build technology-based therapies for dementia carers have increased during the past decade \citep{six}. There are also an increasing number of smartphone applications that cater to the requirements of dementia patients and/or carers \citep{seven}. Cognitive therapies using computers were shown to be more effective than noncomputer-based ones in enhancing cognition in adults with dementia, according to the systematic review by \cite{four}. For dementia caregivers, the effectiveness of technology-based therapies in enhancing psychosocial outcomes was widely observed, but not in increasing caring skills or care self-efficacy, according to a systematic evaluation of these programmes. Patients with dementia and their carers really aren't well-versed in the usage of chatbots \citep{six}.

\section{Chatbot Technology and Health}

Websites, smartphones (like Siri), applications for smartphones and tablets, SMS texting, and even smart home devices (like Alexa) may all be used to run chatbots. It was determined by \cite{lit.ch1} that chatbots may take one of three routes in handling conversations: There are three main types of chatbots: (1) finite state, where the user is guided through a set sequence of dialogue steps; (2) frame based, in which the chatbot starts asking questions and the user's answers assist the chatbot through an unstructured flow of communication; and (3) agent based, in which artificial intelligence enables the chatbot and user to interact through sophisticated conversation. According to the same evaluation, chatbot software may let the platform or the individual user take the lead in a conversation, which may take place either in written or spoken form.

Despite chatbots' widespread application in fields as diverse as customer service, education, website user support, and entertainment, researchers have recently discovered that they may also serve important roles in health care \citep{lit.ch2}, particularly in symptom self-assessment and telemedicine \citep{lit.ch3}. A variety of chatbots have been created and tested to help with issues in health care, such as those related to HIV/AIDS \citep{lit.ch4}, drug misuse and mental health assessment \citep{lit.ch5}, and weight management \citep{lit.ch6}. So far, studies using chatbots in the medical field have shown encouraging outcomes. Psychotherapy and self-cohesion to treatment are two areas where chatbots have been shown to be helpful in an evaluation of their use in psychiatric care \citep{lit.ch7}.

Using chatbots in healthcare has been shown to be beneficial for both patients and healthcare systems. Patients may benefit from the use of chatbots for a variety of reasons, including assistance with therapy management and encouragement at difficult times \citep{lit.ch8}. Telehealth systems may save money and enhance patient outcomes by deploying well-designed chatbots for data collection, education, engagement, and resource allocation \citep{lit.ch3}. \citet{lit.ch1} recently conducted a comprehensive review on the topic and found that despite the promising future of conversational bots in healthcare, only a small amount of research has been conducted on the topic.

\section{Natural Language Processing and Knowledge Base}
It is generally agreed upon that the chatbot engine, also known as the Natural Language Understanding (NLU) engine, is one of the most essential components of a chatbot \citep{lit.kn1}. The Natural Language Understanding (NLU) is responsible for the translation of conversational dialogues into actions that can be comprehended by the computer. In order to comprehend the natural language utilised in interactive user interfaces such as chatbots, NLU engines make use of a wide range of artificial intelligence (AI) techniques. Machine Learning (ML) and Natural Language Processing (NLP) are the two techniques that make up these methodologies \citep{lit.kn1}. The chatbot's ability to understand the context of a user's message is crucial for responding to questions and resolving issues it would otherwise be unable to handle. This is because the user input is ambiguous or might be interpreted in several ways. So when context is retrieved as well as the proper intent is coupled to carry out the intended action for the user, the chatbot is said to have the ability to maintain its state, which is also known as the quantity of user provided input (utterances). Conversational interfaces rely heavily on intents.

It is common for knowledge-based systems to comply with the principles of developing a knowledge base, constructing an inductive reasoning, and a method for user engagement. In order to create computational inference rules that closely approximate human thinking, such solutions need the involvement of subject matter experts in the relevant fields. In an effort to better understand the course of multiple sclerosis, \cite{lit4} built a fuzzy logic-based decision system. Language variables, values, and membership functions were used to quantify the domain knowledge and portray it with the help of clinical professionals. A metaheuristic optimization model was then used to find the maximum possible rate of proper categorization for the healthcare stance process, which was then codified into if-else rules \citep{lit4}. Meningitis diagnostics may be aided by a system that uses a graphical model to express domain knowledge and fuzzy interactions between ideas \citep{lit5}.

\section{Online Discussion Forum on Healthcare}
Back in the day, when the internet was still young, a significant amount of study was conducted to determine how'safe' the health information that was readily accessible was. In spite of persistent worries, researchers discovered minimal levels of reported injury \citep{lit.fo1}. As social media became more popular, the goals of research switched to investigate how people were utilising Internet discussion boards \citep{lit.fo2}. The study came to the conclusion that contact with other people who had a similar health condition was essential for those who had such conditions \citep{lit.fo3}. They were aware of the need to evaluate the information that was being offered, despite the fact that people valued the expertise which others dealing with the same disease may contribute \citep{lit.fo4}. The recent study done by \cite{lit.fo5} shown that social media was extensively utilised by patients as well as carers, with a variety of sites and online forums being used to offer support for patients and caregivers.

People are encouraged to talk about their health on online discussion forums, and there is a forum for almost any imaginable health issue \citep{lit.fo6}. Many healthcare-related studies that use internet forums to collect data on patient experiences do so by doing retrospective or tertiary evaluations of archived postings, which prevents researchers from engaging in follow-up questioning \citep{lit.fo7, lit.fo8}. While more and more health researchers are using online discussion forums to obtain qualitative information from patients \citep{lit.fo9, lit.fo10}, few have examined how effective discussion platforms are for knowledge exchange and virtual communities among health professionals \citep{lit.fo12}.


\def\baselinestretch{1.66}
\medskip


%%% ----------------------------------------------------------------------
